\documentclass[twoside, a4paper, 10pt]{amsart}
\title[ ]{Galois Theory notes}
%\usepackage{amsaddr}
%\email{Kamil.Bulinski@minetec.com.au}
\usepackage{amsfonts}
\usepackage{amsthm}
\usepackage{verbatim}
\usepackage{amsmath, amssymb}
\usepackage{tikz}
\usetikzlibrary{matrix, arrows}
\usepackage{listings}
\usepackage{color}
\usepackage{listings}
\usepackage[all]{xy}
\usepackage[pdftex,colorlinks,linkcolor=blue,citecolor=blue]{hyperref}
\usepackage{graphicx}
\usepackage{float}
\usepackage[margin=3cm]{geometry}
\usepackage{bigints}
\usepackage{dsfont}
\setlength{\textwidth}{6.5in}
\setlength{\oddsidemargin}{0in}
\setlength{\evensidemargin}{0in}
\setlength{\parindent}{0pt}
\setlength{\parskip}{1ex plus 0.5ex minus 0.2ex}
\linespread{1.3}


\setcounter{secnumdepth}{4}

\begin{document}
\maketitle
\raggedbottom


%% Mathcal large
\newcommand{\cA}{\mathcal{A}}
\newcommand{\cB}{\mathcal{B}}
\newcommand{\cC}{\mathcal{C}}
\newcommand{\cD}{\mathcal{D}}
\newcommand{\cE}{\mathcal{E}}
\newcommand{\cF}{\mathcal{F}}
\newcommand{\cG}{\mathcal{G}}
\newcommand{\cH}{\mathcal{H}}
\newcommand{\cI}{\mathcal{I}}
\newcommand{\cJ}{\mathcal{J}}
\newcommand{\cK}{\mathcal{K}}
\newcommand{\cL}{\mathcal{L}}
\newcommand{\cM}{\mathcal{M}}
\newcommand{\cN}{\mathcal{N}}
\newcommand{\cO}{\mathcal{O}}
\newcommand{\cP}{\mathcal{P}}
\newcommand{\cQ}{\mathcal{Q}}
\newcommand{\cR}{\mathcal{R}}
\newcommand{\cS}{\mathcal{S}}
\newcommand{\cT}{\mathcal{T}}
\newcommand{\cU}{\mathcal{U}}
\newcommand{\cV}{\mathcal{V}}
\newcommand{\cW}{\mathcal{W}}
\newcommand{\cX}{\mathcal{X}}
\newcommand{\cY}{\mathcal{Y}}
\newcommand{\cZ}{\mathcal{Z}}
%% Mathbb large
\newcommand{\bA}{\mathbb{A}}
\newcommand{\bB}{\mathbb{B}}
\newcommand{\bC}{\mathbb{C}}
\newcommand{\bD}{\mathbb{D}}
\newcommand{\bE}{\mathbb{E}}
\newcommand{\bF}{\mathbb{F}}
\newcommand{\bG}{\mathbb{G}}
\newcommand{\bH}{\mathbb{H}}
\newcommand{\bI}{\mathbb{I}}
\newcommand{\bJ}{\mathbb{J}}
\newcommand{\bK}{\mathbb{K}}
\newcommand{\bL}{\mathbb{L}}
\newcommand{\bM}{\mathbb{M}}
\newcommand{\bN}{\mathbb{N}}
\newcommand{\bO}{\mathbb{O}}
\newcommand{\bP}{\mathbb{P}}
\newcommand{\bQ}{\mathbb{Q}}
\newcommand{\bR}{\mathbb{R}}
\newcommand{\bS}{\mathbb{S}}
\newcommand{\bT}{\mathbb{T}}
\newcommand{\bU}{\mathbb{U}}
\newcommand{\bV}{\mathbb{V}}
\newcommand{\bW}{\mathbb{W}}
\newcommand{\bX}{\mathbb{X}}
\newcommand{\bY}{\mathbb{Y}}
\newcommand{\bZ}{\mathbb{Z}}


\newcounter{dummy} \numberwithin{dummy}{section}

\theoremstyle{definition}
\newtheorem{mydef}[dummy]{Definition}
\newtheorem{prop}[dummy]{Proposition}
\newtheorem{corol}[dummy]{Corollary}
\newtheorem{thm}[dummy]{Theorem}
\newtheorem{lemma}[dummy]{Lemma}
\newtheorem{eg}[dummy]{Example}
\newtheorem{notation}[dummy]{Notation}
\newtheorem{remark}[dummy]{Remark}
\newtheorem{claim}[dummy]{Claim}
\newtheorem{Exercise}[dummy]{Exercise}
\newtheorem{question}[dummy]{Question}
\newtheorem{conjecture}[dummy]{Conjecture}

\section{Splitting fields and Normal extensions}

\begin{prop} Let $K \leq L$ be fields. Suppose that $\alpha \in L$ is algebraic over $K$ and let $p(x) \in K[x]$ be a minimal polynomial for $\alpha$. Then there is a unique isomorphism $K[x]/(p(x)) \to K[\alpha] = K(\alpha)$ mapping $x$ to $\alpha$ and fixing $K$. \end{prop}

\begin{proof} There is a unique map $K[x] \to K[\alpha]$ mapping $x$ to $\alpha$ and fixing $K$. It is surjective and its kernel is the ideal generated by $p(x)$. \end{proof}

If $\sigma: K \to L$ is a homomorphism of fields and $f = a_0 + a_1x + \cdots + a_nx^n \in F[x]$, then we let $f^{\sigma} = \sigma(a_0) + \sigma(a_1) x + \cdots + \sigma(a_n)x^n \in F[x]$.

\begin{lemma} Suppose that $\sigma: K \to L$ is an isomorphism of fields and suppose $K' = K[\alpha]$ is an extension of $K$ where $f \in F[x]$ the minimal polynomial of $\alpha \in K'$. Let $\sigma:K \to L$ be a field homomorphism.

\begin{itemize}
	\item If $\sigma': K' \to L$ extends $\sigma$, then $f^{\sigma}(\sigma(\alpha')) = 0$
	\item If $\beta \in L$ satisfies that $f^{\sigma}(\beta) = 0$, then there is precisely one extension of $\sigma$ mapping $\alpha$ to $\beta$.
\end{itemize}
\end{lemma}

\begin{proof} The first point is obvious. For the second point, let $\phi:K[x] \to L$ be given by $\phi(P) = P^{\sigma}(\beta)$. This is a ring homomorphism. Now observe that $\phi(f) = f^{\sigma}(\beta) = 0$, thus $\phi$ vanishes on the ideal generated by $f$ and so there is a well defined field homomorpism $\phi:K[x]/(f) \to L$ mapping $x + (f)$ to $\beta$. Finally, we use the isomorphism $ K' \cong K[x]/(f)$ that maps $\alpha$ to $x$ and fixes $K$, giving the desired extension. The extension is clearly unique as $K' = K(\alpha)$.

\end{proof}

\begin{prop} \label{prop: extend morphism to K'} Let $K \leq K'$ be an algebraic field extension and suppose that $\sigma:K \to L$ is a field homomorphism where $L$ is algebraically closed. Then there exists an extension $\sigma':K' \to L$. Moreover, $\sigma'$ must be an isomorphism if $K'$ is algebraically closed and $L$ is algebraic over $\sigma(K)$.

\end{prop}

\begin{proof} Use Zorn's lemma to construct a maximal subfield $K'' \subset K$ such that $\sigma$ extends to $K''$. If $K'' \neq K'$ then choose $\alpha \in K' \setminus K''$. Now as $K'$ ia algebraic over $K$ we can let $f \in K[x]$ be a minimal polynomial of $\alpha$ over $K$. Now as $f^{\sigma}$ has a root in $L$ as $L$ is algebraically closed, we can use the previous lemma to extend $\sigma$ to $K''[\alpha]$, contradicting the maximality of $K''$. If $K'$ is algebriacally closed, then so is $\sigma'(K')$ since any element of $\sigma'(K')[x]$ is of the form $f^{\sigma'}$ for some $f \in K'[x]$ and so we can let $\alpha$ be a root of $f$, giving that $\sigma'(\alpha)$ is a root of $f^{\sigma'}$. Now $\sigma'(K') \geq \sigma(K)$ so if $L$ is algebraical over $\sigma(K)$, then $L$ is also algebraic over $\sigma'(K')$. So if $L$ is algebraically closed then $L = \sigma'(K')$, giving that $\sigma'$ is surjective and thus an isomorphism (all field isomorphisms are injective). \end{proof}

\begin{corol} The algebraic closure of a field $K$ is unique upto an isomorphism fixing $K$.

\end{corol}

\begin{mydef}[Splitting field] Let $K \leq L$ be fields and let $\mathcal{F} \subset K[x]$ be a family of polynomials. We say that $L$ is a splitting field for $\mathcal{F}$ over $F$ if each $f \in \mathcal{F}$ splits into linear factors in $L[x]$ and $L$ is the field generated by $K$ and the roots of all polynomials in $\mathcal{F}$.

\end{mydef}

\begin{prop} A splitting field is unique upto an isomorphism fixing $F$.

\end{prop}

\begin{proof} Let $L \geq K$ and $L' \geq K$ be two splitting fields for a family $\mathcal{F} \subset K[x]$. We note that $L'$ and $L$ are both algebraic over $K$ (as they are generated by roots). This means that we may use Proposition~\ref{prop: extend morphism to K'} to extend the identity map $K \to K$ to a field homomorphism $\sigma:L \to \widehat{L'}$ where $\widehat{L'} \geq L'$ is algebraically closed. However, note that $\sigma(L) \subset L'$ since $\sigma$ maps each root of some $f \in \mathcal{F}$ to a root of $f$ (as $\sigma$ fixes $K$). So $\sigma:L \to L'$ is a homomorphism. It remains to show that $\sigma$ is surjective. To see this, let $f \in \mathcal{F}$ and write $f(x) = \prod_i (x - \alpha_i)$ where $\alpha_i \in L$. Then $f = f^{\sigma} = \prod_i(x - \sigma(\alpha_i))$. This shows that any root in $L'$ of any $f \in \mathcal{F}$ is in the image of $\sigma$ (using the unique factorization property). Thus as $L'$ is generated by these roots, the surjectivity of $\sigma$ follows. \end{proof}

If $K_1$ and $K_2$ are two fields with a common subfield $K$, we say that a homomorphism $K_1 \to K_2$ is a $K$-homomorphism if it restricts to the identity on $K$.

\begin{thm} Let $L$ be an algebraic extension of a field $K$. Then the following are equivalent.

\begin{enumerate}
	\item $L$ is a splitting field for some family of polynomials in $K[x]$.
	\item Any $K$-homomorphism $L \to \overline{L}$, where $\overline{L} \geq L$ is an algebriac closure, restricts to an automorphism of $L$
	\item Any irreducible polynomial in $K[x]$ that has a root in $L$ must decompose into linear factors in $L[x]$.
\end{enumerate}

\end{thm}

\begin{proof} (i) $\implies$ (ii): If $L$ is a splitting field for some polynomials in $K[x]$ and $\sigma:L \to \overline{L}$ is a $K$-homomorphism, then as in the proof of the uniqueness of splitting fields above, we see that $\sigma$ maps into $L$. We also saw that it permutes the roots of a polynomial in $K[x]$ in $L$ and thus the image of $\sigma$ is $L$, thus $\sigma$ is surjective and hence an automorphism.

(ii) $\implies$ (iii): Suppose $f \in K[x]$ is irreducible and has a root $\alpha \in L$. Now if $\alpha' \in \overline{L}$ is another root of $f$, then since $f$ is irreducible we have an isomorphism $ K[\alpha] \to K[\alpha']$ mapping $\alpha$ to $\alpha'$, which we may extend to an $K$-homomorphism $\sigma:L \mapsto \overline{L}$ by a previous Lemma. By condition $(ii)$, we see that $\sigma$ maps $L$ to $L$ and thus $\alpha' = \sigma(\alpha) \in L$. Hence $L$ contains all the roots of $f$.

(iii) $\implies$ (ii): As $L$ is algebraic, every element $\alpha \in L$ is the root of some irreducible polynomial $f \in K[x]$. We thus let $\mathcal{F} \subset K[x]$ be those irreducible polynomials with at least one root in $L$, which split into linear factors by assumption. Thus $L$ is the splitting field of $\mathcal{F}$ over $K$.

\end{proof}

\begin{mydef} We say that an extension $K \leq L$ is normal if it is the splitting field of some family of polynomials.

\end{mydef}

\begin{eg} The extension $\bQ \leq \bQ[2^{1/3}]$ is not normal. To see this we use the characterization (iii) in the Theorem as follows: The polynomial $x^3 - 2$ is irreducible, has one root $2^{1/3}$ in our extension but not any other. Alternatively, we can use (ii) by noting that although there is $\bQ$-homomorphism $\bQ[2^{1/3}] \to \overline{\bQ}$ mapping $2^{1/3}$ to $2^{1/3} e^{2\pi i /3}$, it does not restrict to an automorphism of $\bQ[2^{1/3}]$.

\end{eg}

\begin{eg} Normal is not transitive. As an example, consider the field extensions $\bQ \leq \bQ[\sqrt{2}] \leq \bQ[2^{1/4}]$. The intermediate field extensions are normal (as they are of degree $2$) but the extension $\bQ \leq \bQ[2^{1/4}]$ is not.

\end{eg}

\begin{mydef} If $L \geq K$ is an algebraic extension, then we say that $L' \leq L \leq K$ is a normal closure of $L \geq K$ if $L' \geq K$ is a normal extension and any $L' \geq L'' \geq K$ such that $L'' \geq K$ is normal must satisfy $L'' = L'$. That is, the normal closure if a minimal normal extension.

\end{mydef}

\begin{prop} \label{prop: normal closure existence} Every algebraic extension $L \geq K$ has a normal closure. More precisely, let $\mathcal{F}$ be the set of all irreducible polynomials in $K[x]$ such that each element of $L \setminus K$ is the root of some $f \in \mathcal{F}$. Then the splitting field of $\mathcal{F}$ is the normal closure of $L \geq K$.

\end{prop}

\begin{proof} Let $\overline{L} \geq L$ be the algebraic closure of $L$. Define $\overline{L} \geq L' \geq L$ to be the splitting field for the family $\mathcal{F} \subset K[x]$ of minimal polynomials for elements of $L$. We claim that $L'$ is the normal closure. Thus suppose that $L \leq L'' \leq L'$ is such that $K \leq L''$ is normal. We must show that $L'' = L'$, and since $L'$ is generated by the roots of elements of $\mathcal{F}$, we must show that any root $\alpha \in L'$ of a polynomial $f \in \mathcal{F}$ is in $L''$. To see this, note that by definition $f$ is a minimal polynomial of some $\alpha' \in L$. There is a $K$-homomorphism $\sigma:K[\alpha'] \to \overline{L}$ mapping $\alpha'$ to $\alpha \in L$. As $L'' \geq L \geq K[\alpha']$, we may extend this $K$-homomorphism to $\sigma:L'' \to \overline{L}$. But by characterization (ii) of the normality of $K \leq L''$, we see that $\sigma$ is an automorphism of $L''$. This means that $\alpha = \sigma(\alpha') \in L''$ as $\alpha' \in L \subset L''$. Thus this shows that $L' \subset L''$, and so $L' = L''$ as required. \end{proof}

\begin{prop} If $K \leq L$ is an algebraic extension and $L \leq L_1, L_2 \leq \overline{L}$ are two normal extensions of $K$, then $L_1 \cap L_2$ is a normal extension of $K$. In particular, if $L_1$ and $L_2$ are both normal closures of $L \geq K$, then $L_1 = L_2$.

\end{prop}

\begin{proof} This follows from characterization (iii): If $f \in K[x]$ is irreducible and has a root in $\alpha \in L_1 \cap L_2$, then $f$ decomposes to linear factors in $L_i[x]$ for $i=1,2$. By uniqueness of factorizations, this means that these linear factors are in $(L_1 \cap L_2)[x]$. \end{proof}

\begin{prop} A normal closure of an algebraic extension $L \geq K$ is unique upto an $L$-automorphism.

\end{prop}

\begin{proof} By the previous construction, we have one such normal closure given by $L[\mathcal{R}]$ where $$\mathcal{R} = \{ r \in \overline{L} ~|~ f(r) = 0 \text{ for some } f \in \mathcal{F} \}$$ where $\mathcal{F} \subset K[x]$ is the set of all irreducible polynomials such that each element of $L$ is the root of some $f \in \mathcal{F}$. We now let $L' \geq L$ be another field such that $L' \geq K$ is the normal closure of $L \geq K$. We now construct an isomorphism $L[\mathcal{R}] \to L'$ which fixes $L$. We extend the inclusion $L \to \overline{L'}$ to an $L$-homomorphism $\sigma:L[\mathcal{R}] \to \overline{L'}$. Note that $L'' = \sigma(L[\mathcal{R}]) = L[\sigma(\mathcal{R})]$ contains $L$ and is the splitting field of $\mathcal{F}$ in $\overline{L'}$ over $K$. Thus $L'$ and $L''$ are subfields of $\overline{L}$ that are normal extensions of $K$ and both contain $L$. Moreover, $L''$ is also a normal closure of $L \geq K$ as it follows the construction given in Proposition~\ref{prop: normal closure existence} (i.e., it is a splitting field of minimal polynomials over $K[x]$ of elements in $L$). By the previous proposition, it follows that $L' = L''$, thus $\sigma$ is an isomorphism. \end{proof}

\section{Seperable extensions}

\begin{lemma} An irreducible polynomial $f \in K[x]$ splits into distinct linear factors in some algebraic closure if and only if $f' = 0$.

\end{lemma}

\begin{proof} By the product rule it follows that if $f(\alpha) = 0$ then $\alpha$ is a repeated root if and only if $f'(\alpha) = 0$. If $f$ is irreducible, has a repeated root $\alpha$ and $f' \neq 0$ then $(X-\alpha) | gcd(f, f') | f$, which contradicts the irreducibility of $f$. \end{proof}

As a consequence, if $char K = 0$ then an irreducible polynomial must split into distinct linear factors.

\begin{mydef} We say that $f \in K[x]$ is seperable if $f$ splits into distinct linear factors in some (hence any) algebraic closure of $K$.

\end{mydef}

\begin{thm} If $char K = p$ and $f \in K[x]$ is irreducible, then each root of $f$ has multiplicity $p^r$ where $r$ is minimal non-negative integer such that $f(x) = g(x^{p^r})$ for some $g \in K[x]$.

\end{thm}

\begin{proof} Write $g(x) = \sum_j c_j x^j$. Since $$g'(x) = \sum_{j} j c_j x^j$$ we observe that $g'(x)$ is not the zero polynomial as follows: If $g'(x) = 0$ then $c_j = 0$ whenever $j$ is not divisible by $p$. From this it follows that $g(x) = \sum_k c_{kp} x^{kp} = h(x^p)$. It now follows that $$f(x) = g(x^{p^r}) = h((x^{p^r})^p) =  h(x^{p^{r+1}}),$$ which contradicts the maximality of $r$. Thus $g'(x) \neq 0$. This means that $g(x) = \prod_i (x - \alpha_i)$ where $\alpha_i$ are distinct. Write $\alpha_i = \beta_i^{p^r}$, which exists in an algebraic closure. Note that the $\beta_i$ must also be distinct. Thus $$f(x) = \prod_i (x^{p^r} - \beta_i^{p^r}) = \prod_i (x- \beta_i)^{p^r},  $$ where the last equality follows from Freshman's dream in characteristic $p$. As the $\beta_i$ are distinct, the proof is complete. \end{proof}

\begin{mydef} If $K \leq L$ is an algebraic field extension then $\alpha \in L$ is called seperable over $K$ if the minimal polynomial is seperable (splits over linear factors in some, hence any, algebraic closure). We say that $K \leq L$ is seperable if all elemnts of $L$ are separable over $K$.

\end{mydef}

Thus from above, in characteristic zero all algebraic extensions are seperable, as all irreducible polynomials are seperable.

\begin{mydef} If $K \leq L$ is an algebraic extension, then we let $$Hom_K(L, \overline{K})$$ denote the set of all $K$-homomorphisms $L \to \overline{K}$. We let $$|L:K|_s = |Hom_K(L, \overline{K})|$$ be the seperable degree of $K \leq L$, which does not depend on the choice of $\overline{K}$.

\end{mydef}

\begin{prop} If $K \leq L \leq M$ are algebraic extensions then there is a bijection $$Hom_K(L, \overline{K}) \times Hom_L(M, \overline{K}) \to Hom_K(M, \overline{K}).$$ In particular $$|M:K|_s = |L:K|_s |M:L|_s.$$

\end{prop}

\begin{proof} For each $\sigma \in Hom_K(L, \overline{K})$ we choose an arbitrary (there are many choices) $\phi(\sigma):\overline{K} \to \overline{K}$ automorphism that extends $\sigma$, where we have used Proposition???. Now we define a mapping $$Hom_K(L, \overline{K}) \times Hom_L(M, \overline{K}) \to Hom_K(M, \overline{K})$$ by $$(\sigma, \tau) \mapsto \phi(\sigma) \circ \tau.$$ Let us first check that it is well defined. If $k \in K$ then $$(\phi(\sigma) \circ \tau) (k) = \phi(\sigma) (\tau(k)) = \phi(\sigma) (k) = \sigma(k) = k,$$ so indeed $\phi(\sigma) \circ \tau$ is a $K$-homomorphism. To show injectivity, suppose that $$\phi(\sigma) \circ \tau = \phi(\sigma') \circ \tau'.$$ Then for any $\ell \in L$ we have that $$\phi(\sigma)(\tau(\ell)) = \phi(\sigma)(\ell) = \sigma(\ell)$$ and by the same arugment $\phi(\sigma')(\tau'(\ell)) = \sigma'(\ell)$. Thus $\sigma = \sigma'$. This means that $\phi(\sigma) = \phi(\sigma')$ and so by injectivity of field automorphisms, we must have that $\tau'(m) = \tau(m)$ for all $m \in M$. So $\tau = \tau'$. It now remains to show injectivity. Thus suppose that $\gamma \in Hom_K(M, \overline{K})$. Let $\sigma$ be the restriction of $\gamma$ to $L$ and observe that $\sigma \in Hom_K(L, \overline{K})$. Now let $$\tau = \phi(\sigma)^{-1} \circ \gamma:M \to \overline{K}.$$ If $\ell \in L$ then $$\tau(\ell) = \phi(\sigma)^{-1}(\gamma(\ell)) = \phi(\sigma)^{-1}(\sigma(\ell)) = \phi(\sigma)^{-1} \phi(\sigma)(\ell) = \ell,$$ thus indeed $\tau \in Hom_L(M, \overline{K})$. This shows that $\gamma = \phi(\sigma) \circ \tau$ is in the image of our map, thus our map is surjective. \end{proof}

\begin{prop} If $K \leq L$ is a finite extension then 
\begin{enumerate}
	\item If $K$ has characteristic zero then $|L:K| = |L:K|_s$
	\item If $K$ has characteristic $p$ then $|L:K| = p^r|L:K|_s$ for some integer $r \geq 0$.
\end{enumerate}
\end{prop}

\begin{proof} By finiteness of this extension $L$ can be obtained from $K$ by finitely many simple extensions, so we only need to prove this when $L = K(\alpha)$ is a simple extension and then use the previous proposition to give the general case by induction. If $Char K = 0$ then we know that $|L:K| = deg f = |L:K|_s$ where $f \in K[x]$ is the minimal polynomial of $\alpha$, where we have used the fact that $f$ is seperable and there is a unique $K$-homomorphism mapping $\alpha$ to any given root of $f$. If $Char K = p$ then $|L:K| = deg f = p^r |L:K|_s$ where $r$ is maximal integer such that $f(x) = g(x^{p^r})$ for some polynomial $g(x) \in K[x]$, as seen in a previously proven result. Thus completing the proof. \end{proof}



\end{document}